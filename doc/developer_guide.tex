% 
% Copyright (c) 2013, William Magato
% All rights reserved.
% 
% Redistribution and use in source and binary forms, with or without
% modification, are permitted provided that the following conditions are met:
% 
%    1. Redistributions of source code must retain the above copyright notice,
%       this list of conditions and the following disclaimer.
% 
%    2. Redistributions in binary form must reproduce the above copyright
%       notice, this list of conditions and the following disclaimer in the 
%       documentation and/or other materials provided with the distribution.
% 
% THIS SOFTWARE IS PROVIDED BY THE COPYRIGHT HOLDER(S) AND CONTRIBUTORS 
% ''AS IS'' AND ANY EXPRESS OR IMPLIED WARRANTIES, INCLUDING, BUT NOT LIMITED 
% TO, THE IMPLIED WARRANTIES OF MERCHANTABILITY AND FITNESS FOR A PARTICULAR 
% PURPOSE ARE DISCLAIMED. IN NO EVENT SHALL THE COPYRIGHT HOLDER(S) OR 
% CONTRIBUTORS BE LIABLE FOR ANY DIRECT, INDIRECT, INCIDENTAL, SPECIAL, 
% EXEMPLARY, OR CONSEQUENTIAL DAMAGES (INCLUDING, BUT NOT LIMITED TO, 
% PROCUREMENT OF SUBSTITUTE GOODS OR SERVICES; LOSS OF USE, DATA, OR PROFITS; 
% OR BUSINESS INTERRUPTION) HOWEVER CAUSED AND ON ANY THEORY OF LIABILITY, 
% WHETHER IN CONTRACT, STRICT LIABILITY, OR TORT (INCLUDING NEGLIGENCE OR 
% OTHERWISE) ARISING IN ANY WAY OUT OF THE USE OF THIS SOFTWARE, EVEN IF 
% ADVISED OF THE POSSIBILITY OF SUCH DAMAGE.
% 
% The views and conclusions contained in the software and documentation are 
% those of the authors and should not be interpreted as representing official 
% policies, either expressed or implied, of the copyright holder(s) or 
% contributors.
% 

\documentclass[draft]{article}

\usepackage[draft=false,colorlinks=true]{hyperref}
\usepackage{parskip}

\begin{document}

\pagenumbering{alph}

\begin{titlepage}
\raggedleft
{\huge{Developer Guide}\\[1.0in]}
{\Huge{\textbf{llamaOS}}\\[0.125in]}
{\Large{% generated content
% do not edit this file
Version 1.1
}}
\vfill
\itshape
Experimental Computing Laboratory\\
University of Cincinnati\\[0.125in]
\today
\end{titlepage}

\pagenumbering{roman}

\tableofcontents
\clearpage

\pagenumbering{arabic}

\section{Introduction}\label{introduction}

When working with llamaOS, computer resources need to be thought of as either 
\emph{development} or \emph{runtime} systems.  A development system is used for 
developing the llamaOS source code and building binary images to be executed on 
a runtime system.  A single system can be used for both development and runtime 
execution provided it fulfills the requirements for both use cases.  This can 
be a convenient way to develop new features and quickly test applications.  A 
cluster of capable runtime systems is necessary for true performance 
benchmarking though.

\subsection{Development System}

The requirements for a development system are:

\begin{itemize}
  \item Version control
    \subitem GitHub compatible Git client
    \subitem GitHub compatible web browser
    \subitem SSH client
  \item Compiler/build tools
    \subitem GNU GCC 4.7.1 or greater (languages: C, C++, Fortran)
    \subitem GNU Make
    \subitem GNU Binutils
    \subitem GNU zip
  \item Code analysis (optional)
    \subitem Cppcheck
  \item Documentation (optional)
    \subitem Doxygen
    \subitem \LaTeX
\end{itemize}

\subsubsection{Version Control}

The source code for the llamaOS project is maintained with the 
\href{http://git-scm.com/}{Git}\footnote{\url{http://git-scm.com/}} 
distributed version control system.  The repository is hosted on 
\href{https://github.com/wilseypa/llamaOS}{GutHub}\footnote{\url{
https://github.com/wilseypa/llamaOS}}, a web-based service with some basic 
project management features.  Refer to the GitHub documentation for details on 
Git client and we browser minimum requirements.

An SSH client is needed to clone and remotely administer the llamaOS Git 
repository.  Most modern Linux distributions arrive with an SSH client 
pre-installed.  If not, refer to the distribution's package management for 
instructions on installing and configuring a client.

\subsubsection{Compiler/Build Tools}

The llamaOS build system is currently designed to compile source code using 
the \href{http://gcc.gnu.org/}{GNU Compiler 
Collection}\footnote{\url{http://gcc.gnu.org/}}, 
\href{http://www.gnu.org/software/make/}{Make}\footnote{\url{
http://www.gnu.org/software/make/}} and various
\href{http://www.gnu.org/software/binutils/}{binary 
tools}\footnote{\url{http://www.gnu.org/software/binutils/}}.  Similar to the 
SSH client, all of these tools are usually pre-installed for most modern Linux 
distributions.  However, the distribution may not have the necessary versions 
available.  llamaOS takes advantage of recent features of GCC and C++11 that 
require version 4.7.1 or greater.  If the desired development platform does not 
support recent versions of GCC, it can be downloaded and installed along side 
the systems native build system.

\href{http://www.gzip.org/}{GNU zip}\footnote{\url{http://www.gzip.org/}} is a 
compression tool used to compress the final binary llamaOS executable image. 
The binary image is compressed to save stoarge space and time when transmitting 
the file to the runtime system(s).

\subsubsection{Code Analysis}

To adhere to software engineering best practices, code analysis and automated 
testing is integrated into the llamaOS build system.  At this time, 
\href{http://cppcheck.sourceforge.net/}{ccpcheck}\footnote{\url{
http://cppcheck.sourceforge.net/}} is used for static analysis of all C/C++ 
files written by the llamaOS contributors.  The 
\href{https://code.google.com/p/googletest/}{Google C++ Testing 
Framework}\footnote{\url{https://code.google.com/p/googletest/}} provides unit 
testing capabilities to the project.  Both code analysis and automated testing 
procedures are still under heavy construction.  Additional tools will be 
evaluated  and incorporated into the llamaOS build system as warranted.  

\subsubsection{Documentation}

In-line source code documentation of llamaOS is produced and distributed using 
\href{http://www.stack.nl/~dimitri/doxygen/}{Doxygen}\footnote{\url{
http://www.stack.nl/~dimitri/doxygen/}}.  The general rule is to document all 
aspects of llamaOS that are publicly accessible from application logic.  System 
level documentation files, such as this manual, are typeset and distributed as 
PDF files using \LaTeX.

\subsection{Runtime System}

The requirements for a runtime system are:

\begin{itemize}
  \item Hypervisor
    \subitem Xen 4.2.0 or greater
  \item Network Hardware (optional, but required for llamaNET)
    \subitem Intel PCI/PCIe NIC (82574 or 80003es2lan chipset)
\end{itemize}

\subsubsection{Hypervisor}

Presently, llamaOS requires 
\href{http://www.xenproject.org/}{Xen}\footnote{\url{http://www.xenproject.org/}
} version 4.2.0 or greater installed on a 64-bit x86 PC based system.  Future 
plans include support for alternative hypervisor technologies and hardware 
architectures but nothing has been implemented at this time.  Users of llamaOS 
must also have system permissions to create and administer Xen virtual 
machines.  This requirement normally implies that the users need to have 
\emph{sudo} rights.  This is a side-effect of using Xen, and has nothing to do 
with llamaOS.

\subsubsection{Network Hardware}

Although llamaOS is capable of executing applications on a single virtual 
machine, it is designed for fine grained, message passing, parallel 
applications.  Virtual machines communicate with each other using a high 
performance data transfer framework called llamaNET.  This framework provides 
communication between virtual machines executing on different physical systems 
using commodity Ethernet adapters.  A llamaNET driver must be written for 
physical hardware components to interact with the llamaNET framework from 
within a llamaOS virtual machine.  Software exists to support Intel PCIe 
network adapters, specifically for the 82574/80003es2lan chipset, already.  
Support for additional hardware can be added at a later time as needed.

\section{Getting and Building the Source}

Once the items specified in section \ref{introduction} are installed on a 
development system, a user can get the source and build the binaries needed 
for application development and execution.

\subsection{GitHub}

Get the llamaOS source code by cloning the Git repository.

\begin{quote}
 \texttt{\$ git clone git@github.com:wilseypa/llamaOS.git}
\end{quote}

After execting this command on a development system, a folder named 
\emph{llamaOS} containing all of the source files under version control will 
appear in the current directory.  Refer to the 
\href{http://git-scm.com/documentation}{Git 
documentation}\footnote{\url{http://git-scm.com/documentation}} and/or 
\emph{man} pages for information pertaining to working with the Git client.  
Also, the Git client provides some documentation by passing \emph{help} as the 
program parameter.

\begin{quote}
 \texttt{\$ git help clone}
\end{quote}

The URL passed to the \emph{clone} command, shown above, can be found on the 
right side of the llamaOS GutHub project page.  llamaOS contributors will need 
to setup a personal GitHub account and request that the llamaOS administer add 
them to the project.  Cloning the repository for the first time will likely 
produce a permissions error concerning SSH keys.  A public key must be uploaded 
into the personal account page of each user.  GitHub has a 
\href{https://help.github.com/articles/generating-ssh-keys}{link}\footnote{\url{
https://help.github.com/articles/generating-ssh-keys }} at the top of the SSH 
settings page with instructions on how to accomplish this task.

Git has many advanced features that make it an extremely usefull development 
tool.  llamaOS contributors should be familiar with basic branching and remote 
repository management at a minimum.  A branching model policy similar to 
\href{http://nvie.com/posts/a-successful-git-branching-model/}{this}
\footnote { \url { http://nvie.com/posts/a-successful-git-branching-model/}} 
may be adopted in the future.

\subsection{Makefiles}

Building the llamaOS project can be as simple as typing \emph{make} at the 
project's root directory.

\begin{quote}
\texttt{llamaOS \$ make}
\end{quote}

The system of llamaOS makefiles in the \emph{build} folder attempt to determine 
what tools are available on the development system and avoid building 
components that can not be supported.  The development system must provide 
support for building the core libraries though.  A common source of error is 
when the system's native compiler does not meet the minimum requirements.

\begin{quote}
\texttt{error: unrecognized command line option '-std=gnu++11'}
\end{quote}

This message indicates that the compiler does not support features of C++11.  
Fortunately, multiple versions of GCC can coexist on a development system.  
Installing an alternative version will allow for llamaOS development at 
minimal risk of causing system configuration issues.  To this this, download 
the source of a version of GCC that meets the minimum requirements for building 
llamaOS descibed in section \ref{introduction}.  Follow the provided 
instructions for building GCC while using the following configuration 
parameters.

\begin{quote}
\texttt{--prefix=/opt/gcc-4.8.0 --enable-languages=c,c++,fortran}
\end{quote}

Replace the \emph{prefix} value with something more appropriate of course.  
Next, add references to the alternative compiler tools to the 
\emph{custom-vars.mk} file.

\begin{quote}
CC = /opt/gcc-4.8.0/bin/gcc\newline
CXX = /opt/gcc-4.8.0/bin/g++\newline
F77 = /opt/gcc-4.8.0/bin/gfortran\newline
F90 = /opt/gcc-4.8.0/bin/gfortran
\end{quote}


\section{System Libraries}

\subsection{glibc}
\subsection{gcc}
\subsection{xen}

\section{llamaOS API}

\section{Network/Device Drivers}

\section{Communication Libraries}

\section{Benchmark/Test Applications}

\end{document}
