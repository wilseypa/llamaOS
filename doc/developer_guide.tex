% 
% Copyright (c) 2013, William Magato
% All rights reserved.
% 
% Redistribution and use in source and binary forms, with or without
% modification, are permitted provided that the following conditions are met:
% 
%    1. Redistributions of source code must retain the above copyright notice,
%       this list of conditions and the following disclaimer.
% 
%    2. Redistributions in binary form must reproduce the above copyright
%       notice, this list of conditions and the following disclaimer in the 
%       documentation and/or other materials provided with the distribution.
% 
% THIS SOFTWARE IS PROVIDED BY THE COPYRIGHT HOLDER(S) AND CONTRIBUTORS 
% ''AS IS'' AND ANY EXPRESS OR IMPLIED WARRANTIES, INCLUDING, BUT NOT LIMITED 
% TO, THE IMPLIED WARRANTIES OF MERCHANTABILITY AND FITNESS FOR A PARTICULAR 
% PURPOSE ARE DISCLAIMED. IN NO EVENT SHALL THE COPYRIGHT HOLDER(S) OR 
% CONTRIBUTORS BE LIABLE FOR ANY DIRECT, INDIRECT, INCIDENTAL, SPECIAL, 
% EXEMPLARY, OR CONSEQUENTIAL DAMAGES (INCLUDING, BUT NOT LIMITED TO, 
% PROCUREMENT OF SUBSTITUTE GOODS OR SERVICES; LOSS OF USE, DATA, OR PROFITS; 
% OR BUSINESS INTERRUPTION) HOWEVER CAUSED AND ON ANY THEORY OF LIABILITY, 
% WHETHER IN CONTRACT, STRICT LIABILITY, OR TORT (INCLUDING NEGLIGENCE OR 
% OTHERWISE) ARISING IN ANY WAY OUT OF THE USE OF THIS SOFTWARE, EVEN IF 
% ADVISED OF THE POSSIBILITY OF SUCH DAMAGE.
% 
% The views and conclusions contained in the software and documentation are 
% those of the authors and should not be interpreted as representing official 
% policies, either expressed or implied, of the copyright holder(s) or 
% contributors.
% 

\documentclass[draft]{article}

\usepackage[draft=false,colorlinks=true]{hyperref}
\usepackage{parskip}

\begin{document}

\pagenumbering{alph}

\begin{titlepage}
\raggedleft
{\huge{Developer Guide}\\[1.0in]}
{\Huge{\textbf{llamaOS}}\\[0.125in]}
{\Large{% generated content
% do not edit this file
Version 1.1
}}
\vfill
\itshape
Experimental Computing Laboratory\\
University of Cincinnati\\[0.125in]
\today
\end{titlepage}

\pagenumbering{roman}

\tableofcontents
\clearpage

\pagenumbering{arabic}

\section{Introduction}

When working with llamaOS, computer resources need to be thought of as either 
\emph{development} or \emph{runtime} systems.  A development system is used for 
developing the llamaOS source code and building binary images to be executed on 
a runtime system.  A single system can be used for both development and runtime 
execution provided it fulfills the requirements for both use cases.  This can 
be a convenient way to develop new features and quickly test applications.  A 
cluster of capable runtime systems is necessary for true performance 
benchmarking though.

\subsection{Development System}

The requirements for a development system are:

\begin{itemize}
  \item Version Control
    \subitem GitHub compatible Git client
    \subitem GitHub compatible web browser
    \subitem SSH client
  \item Compiler/Build Tools
    \subitem GNU GCC 4.7.1 or greater (languages: C, C++, Fortran)
    \subitem GNU Binutils
    \subitem GNU gzip
  \item Code Analysis (optional)
    \subitem Cppcheck
  \item Documentation (optional)
    \subitem Doxygen
    \subitem \LaTeX
\end{itemize}

\subsubsection{Version Control}

The source code for the llamaOS project is maintained with the 
\href{http://git-scm.com/}{Git}\footnote{\url{http://git-scm.com/}} 
distributed version control system.  The repository is hosted on 
\href{https://github.com/wilseypa/llamaOS}{GutHub}\footnote{\url{
https://github.com/wilseypa/llamaOS}}, a web-based service with some basic 
project management features.  Refer to the GitHub documentation for details on 
minimum Git client and we browser requirements.

An SSH client is needed to clone and remotely administer the llamaOS Git 
repository.  Most modern Linux distributions arrive with an SSH client 
pre-installed.  If not, refer to the distribution's package management for 
instructions on installing a client.

\subsubsection{Compiler/Build Tools}

The llamaOS build system is currently designed to compile source code using 
the \href{http://gcc.gnu.org/}{GNU Compiler 
Collection}\footnote{\url{http://gcc.gnu.org/}} and 
\href{http://www.gnu.org/software/binutils/}{binary 
tools}\footnote{\url{http://www.gnu.org/software/binutils/}}.  Similar to the 
SSH client, GNU GCC is usually pre-installed for most modern Linux 
distributions.  However, the distribution may not have the necessary versions 
available.  llamaOS takes advantage of recent features of GCC and C++x11 that 
required version 4.7.1 or greater.  If the desired development platform does not 
support recent versions of GCC, it can be downloaded and installed along side 
the systems native build system.

GNU gzip is a compression tool used to build the final binary llamaOS 
executable image. The binary is compressed to save space and time transmitting 
the file to the runtime system for execution.

\subsubsection{Code Analysis}

To adhere to software engineering best practices, code analysis and automated 
testing is integrated into the llamaOS build system.  At this time, 
\href{http://cppcheck.sourceforge.net/}{ccpcheck}\footnote{\url{
http://cppcheck.sourceforge.net/}} is used for static analysis of all C/C++ 
files written by the llamaOS team.  Additional tools will be evaluated and 
incorporated into the llamaOS build system when warranted.  The 
\href{https://code.google.com/p/googletest/}{Google C++ Testing 
Framework}\footnote{\url{https://code.google.com/p/googletest/}} provides unit 
testing capabilities to the project.  Both code analysis and automated testing 
procedures are still under heavy construction and at this time.

\subsubsection{Documentation}

In-line source code documentation of llamaOS is produced and distributed using 
\href{http://www.stack.nl/~dimitri/doxygen/}{Doxygen}\footnote{\url{
http://www.stack.nl/~dimitri/doxygen/}}.  The general rule is to document all 
aspects of llamaOS that are publicly accessible from application logic.  System 
level documentation files, such as this manual, are typeset and distributed as 
PDF files using \LaTeX.

\subsection{Runtime System}

The requirements for a development system are:

\begin{itemize}
  \item Hypervisor
    \subitem Xen 4.2.0 or greater
  \item Network Hardware (optional, but required for llamaNET)
    \subitem Intel PCI/PCIe NIC (82574 or 80003es2lan chipset)
\end{itemize}

\subsubsection{Hypervisor}

\subsubsection{Network Hardware}

\section{Source Repository}





\url{https://help.github.com/articles/generating-ssh-keys}


\section{Build System}

\section{System Libraries}

\section{llamaOS API}

\section{Network/Device Drivers}

\section{Communication Libraries}

\section{Benchmark/Test Applications}

\end{document}
